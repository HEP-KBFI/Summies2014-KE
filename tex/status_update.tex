\documentclass[12pt,a4paper]{article}

% paketid
\usepackage[utf8]{inputenc}
\usepackage[english]{babel}
\usepackage{fullpage}
\usepackage[raggedright]{titlesec}
\usepackage[titles,subfigure]{tocloft}
\usepackage{hyperref}
\usepackage[svgnames]{xcolor}
\usepackage{mathtools}
\usepackage{float}
\usepackage{graphicx}
\usepackage[titletoc]{appendix}
\usepackage{enumerate}
\usepackage{times}
\usepackage{titlesec}
\usepackage[shortlabels]{enumitem}
\usepackage[bottom]{footmisc}
\usepackage[margin=0.6in]{geometry}
\usepackage{multicol}
\usepackage[utopia]{mathdesign}
\usepackage[OMLmathrm,OMLmathbf]{isomath}
\usepackage{subfigure}
\usepackage{braket}
\def\pdfshellescape{1}
\usepackage{epstopdf}
\usepackage{comment}
\usepackage{wrapfig}
\usepackage{listings}
\usepackage{setspace}
%\usepackage[backend=bibtex,sorting=none,style=ieee]{biblatex}
\usepackage[multidot]{grffile}

%\addbibresource{sources.bib}

\definecolor{Code}{rgb}{0,0,0}
\definecolor{Decorators}{rgb}{0.5,0.5,0.5}
\definecolor{Numbers}{rgb}{0.5,0,0}
\definecolor{MatchingBrackets}{rgb}{0.25,0.5,0.5}
\definecolor{Keywords}{rgb}{0,0,1}
\definecolor{self}{rgb}{0,0,0}
\definecolor{Strings}{rgb}{0,0.63,0}
\definecolor{Comments}{rgb}{0,0.63,1}
\definecolor{Backquotes}{rgb}{0,0,0}
\definecolor{Classname}{rgb}{0,0,0}
\definecolor{FunctionName}{rgb}{0,0,0}
\definecolor{Operators}{rgb}{0,0,0}
\definecolor{Background}{rgb}{0.98,0.98,0.98}

\lstnewenvironment{python}[1][]{
\lstset{
%numbers=left,
%numberstyle=\footnotesize,
%numbersep=1em,
xleftmargin=1em,
framextopmargin=2em,
framexbottommargin=2em,
showspaces=false,
showtabs=false,
showstringspaces=false,
frame=l,
tabsize=4,
% Basic
basicstyle=\ttfamily\footnotesize\setstretch{1},
%backgroundcolor=\color{Background},
language=Python,
% Comments
commentstyle=\color{Comments}\slshape,
% Strings
stringstyle=\color{Strings},
morecomment=[s][\color{Strings}]{"""}{"""},
morecomment=[s][\color{Strings}]{'''}{'''},
% keywords
morekeywords={import,from,class,def,for,while,if,is,in,elif,else,not,and,or,print,break,continue,return,True,False,None,access,as,,del,except,exec,finally,global,import,lambda,pass,print,raise,try,assert},
keywordstyle={\color{Keywords}\bfseries},
% additional keywords
morekeywords={[2]@invariant},
keywordstyle={[2]\color{Decorators}\slshape},
emph={self},
emphstyle={\color{self}\slshape},
%
}}{}

%\linespread{1.5} % reavahe
\titleformat{\chapter}[hang]{\bfseries\Large}{\thechapter.}{2pc}{} % teeb pealkirjad korda
\titlelabel{\thetitle.\quad}
\titlespacing*{\chapter}{0pt}{-50pt}{20pt}
\setlist[enumerate]{itemsep=0mm}

\definecolor{dark-red}{rgb}{0.4,0.15,0.15}

\hypersetup{
    colorlinks, linkcolor={Black},
    citecolor={Black}, urlcolor={dark-red}
}

\renewcommand*\d{\mathop{}\!\mathrm{d}}

\DeclareGraphicsExtensions{.pdf,.png,.jpg,.eps}

%\setlength\parindent{0pt} % noindent by default

\newcommand*{\TitleFont}{%
      \usefont{\encodingdefault}{\rmdefault}{b}{n}%
      \fontsize{16}{20}%
      \selectfont}

%\DeclareMathAlphabet{\pazocal}{OMS}{zplm}{m}{n} % teistsugune kalligraafiline täht, nt \pazocal{L}

\epstopdfsetup{outdir=./converted_pdf/}

\title{\TitleFont\textbf{Current status of the project\\
,,Improved background estimation for the ttH analysis''}}
\author{Karl Ehatäht}
\date\today

\begin{document}
\maketitle

\section{Description}

So far I have done the following
\begin{itemize}
\item created histograms from the tt+jets events for specific ranges of jet $p_t$, $\eta$ and flavor
\item normalized the resulting histograms in order to treat them as PDFs
\item plotted b-tagging efficiency, and mistagged c- and light jets as a function of the CSV cutting point
\item sampled the histograms once
\item did statistical tests between the original histograms and the sampled ones
\item wrote a program which generates a CSV value until the event passes the given working point
\item visualize all the results
\end{itemize}

\newpage
\section{Efficiency plots}
\begin{figure}[H]
\subfigure{
	\centering
	\includegraphics[scale=.45]{../plots/eps/effs_[20,30]_[0,0.8].eps}
	\label{fig:effs_[20,30]_[0,0.8]}
}
\subfigure{
	\centering
	\includegraphics[scale=.45]{../plots/eps/effs_[20,30]_[0.8,1.6].eps}
	\label{fig:effs_[20,30]_[0.8,1.6]}
}

\subfigure{
	\centering
	\includegraphics[scale=.45]{../plots/eps/effs_[20,30]_[1.6,2.5].eps}
	\label{fig:effs_[20,30]_[1.6,2.5]}
}
\subfigure{
	\centering
	\includegraphics[scale=.45]{../plots/eps/effs_[30,40]_[0,0.8].eps}
	\label{fig:effs_[30,40]_[0,0.8]}
}

\subfigure{
	\centering
	\includegraphics[scale=.45]{../plots/eps/effs_[30,40]_[0.8,1.6].eps}
	\label{fig:effs_[30,40]_[0.8,1.6]}
}
\subfigure{
	\centering
	\includegraphics[scale=.45]{../plots/eps/effs_[30,40]_[1.6,2.5].eps}
	\label{fig:effs_[30,40]_[1.6,2.5]}
}

\subfigure{
	\centering
	\includegraphics[scale=.45]{../plots/eps/effs_[40,60]_[0,0.8].eps}
	\label{fig:effs_[40,60]_[0,0.8]}
}
\subfigure{
	\centering
	\includegraphics[scale=.45]{../plots/eps/effs_[40,60]_[0.8,1.6].eps}
	\label{fig:effs_[40,60]_[0.8,1.6]}
}
\end{figure}
\begin{figure}[H]
\subfigure{
	\centering
	\includegraphics[scale=.45]{../plots/eps/effs_[40,60]_[1.6,2.5].eps}
	\label{fig:effs_[40,60]_[1.6,2.5]}
}
\subfigure{
	\centering
	\includegraphics[scale=.45]{../plots/eps/effs_[60,100]_[0,0.8].eps}
	\label{fig:effs_[60,100]_[0,0.8]}
}

\subfigure{
	\centering
	\includegraphics[scale=.45]{../plots/eps/effs_[60,100]_[0.8,1.6].eps}
	\label{fig:effs_[60,100]_[0.8,1.6]}
}
\subfigure{
	\centering
	\includegraphics[scale=.45]{../plots/eps/effs_[60,100]_[1.6,2.5].eps}
	\label{fig:effs_[60,100]_[1.6,2.5]}
}

\subfigure{
	\centering
	\includegraphics[scale=.45]{../plots/eps/effs_[100,160]_[0,0.8].eps}
	\label{fig:effs_[100,160]_[0,0.8]}
}
\subfigure{
	\centering
	\includegraphics[scale=.45]{../plots/eps/effs_[100,160]_[0.8,1.6].eps}
	\label{fig:effs_[100,160]_[0.8,1.6]}
}

\subfigure{
	\centering
	\includegraphics[scale=.45]{../plots/eps/effs_[100,160]_[1.6,2.5].eps}
	\label{fig:effs_[100,160]_[1.6,2.5]}
}
\subfigure{
	\centering
	\includegraphics[scale=.45]{../plots/eps/effs_[160,inf]_[0,0.8].eps}
	\label{fig:effs_[160,inf]_[0,0.8]}
}
\end{figure}
\begin{figure}[H]

\subfigure{
	\centering
	\includegraphics[scale=.45]{../plots/eps/effs_[160,inf]_[0.8,1.6].eps}
	\label{fig:effs_[160,inf]_[0.8,1.6]}
}
\subfigure{
	\centering
	\includegraphics[scale=.45]{../plots/eps/effs_[160,inf]_[1.6,2.5].eps}
	\label{fig:effs_[160,inf]_[1.6,2.5]}
}
\end{figure}

\end{document}
















