\documentclass[12pt,a4paper]{article}

% paketid
\usepackage[utf8]{inputenc}
\usepackage[english]{babel}
\usepackage{fullpage}
\usepackage[raggedright]{titlesec}
\usepackage[titles,subfigure]{tocloft}
\usepackage{hyperref}
\usepackage[svgnames]{xcolor}
\usepackage{mathtools}
\usepackage{float}
\usepackage[pdftex]{graphicx}
\usepackage[titletoc]{appendix}
\usepackage{enumerate}
\usepackage{calrsfs}
\usepackage{times}
\usepackage{titlesec}
\usepackage[letterspace=350]{microtype} % no hyphenation anymore
\usepackage[shortlabels]{enumitem}
\usepackage[bottom]{footmisc}
\usepackage[margin=0.85in]{geometry}
\usepackage{multicol}
\usepackage{subfigure}
\usepackage{braket}
\def\pdfshellescape{1}
\usepackage{epstopdf}
\usepackage{wrapfig}
\usepackage{setspace}
\usepackage[multidot]{grffile}
\usepackage{feynmp} % Feynman diagrams
\usepackage[backend=bibtex,bibencoding=utf8,sorting=none]{biblatex}

\addbibresource{sources.bib}

%\linespread{1.5} % reavahe
\titleformat{\chapter}[hang]{\bfseries\Large}{\thechapter.}{2pc}{} % teeb pealkirjad korda
\titlelabel{\thetitle.\quad}
\titlespacing*{\chapter}{0pt}{-50pt}{20pt}
\titlespacing*{\section}{0pt}{-15pt}{5pt}
\titlespacing*{\subsection}{0pt}{0pt}{0pt}
\setlist[enumerate]{itemsep=0mm}

\subfigcapskip = 20pt

\definecolor{dark-red}{rgb}{0.4,0.15,0.15}

\hypersetup{
	colorlinks=true,
    linkcolor={Black},
    citecolor={Black}, urlcolor={dark-red}
}

\newcommand{\mcell}[2][c]{
	\begin{tabular}[#1]{@{}c@{}}#2\end{tabular}
}

\DeclareMathAlphabet{\pazocal}{OMS}{zplm}{m}{n}

\DeclareGraphicsExtensions{.pdf,.png,.jpg,.eps}
\DeclareGraphicsRule{*}{mps}{*}{}

%\setlength\parindent{0pt} % noindent by default

\newcommand*{\TitleFont}{%
      \usefont{\encodingdefault}{\rmdefault}{b}{n}%
      \fontsize{16}{20}%
      \selectfont}

%\DeclareMathAlphabet{\pazocal}{OMS}{zplm}{m}{n} % teistsugune kalligraafiline täht, nt \pazocal{L}

\epstopdfsetup{outdir=./converted_pdf/}

\title{\TitleFont\textbf{Report on the Summer Student project\\,,Improved background estimation for the $t\bar{t}H$ analysis''}}
\author{Karl Ehatäht\\[0.5em]
\textit{Supervisors: Lorenzo Bianchini, Mario Kadastik, Joosep Pata}}
\date\today

\begin{document}
\maketitle

A direct probe of the Yukawa coupling of Higgs boson to top quark can be measured in only those processes where top quark radiates a Higgs boson (see Fig. \ref{fig:ttH}).
The process, denoted as $t\bar{t}H$, has the largest cross section when the Higgs decay mode, $H \to b\bar{b}$, is the most dominant \cite{dittmaier2011handbook}.
The large multijet background in the final state  can be reduced by selecting only the events with one or two charged isolated leptons.
However, the signal is still plauged by QCD production of $t\bar{t}+b\bar{b}$, $t\bar{t}+c\bar{c}$ and $t\bar{t}+jj$ after the cut \cite{ATLASconference}.
Although the former subprocess, $pp\to t\bar{t}b\bar{b}$, remains still irreducible, the rest can be suppressed by requiring the jets to be b-tagged.
In other words, a real number (discriminator, e.g. CSV) is calculated (based on the jet kinematics and information about secondary vertex) and then assigned to each jet which is used as a boolean variable to separate $b$-jets from non-$b$ jets, as it is usually carried out in standard analysis.
Since the distribution of discriminator is known from Monte Carlo (MC) simulations, it can be used probabilistically to reweigh the events.
The method rescues all MC events instead of throwing most of them away.
Intrinsically, as it turns out, the method cannot be applied to data from measurements.
\vspace*{1em}
\begin{figure}[H]
	\begin{center}
	\subfigure[t-channel gluon fusion.]{
		\centering
		\begin{fmffile}{fgraphs}
			\begin{fmfgraph*}(160,80)
				\fmfleft{i1,i2,i3}
				\fmfright{o1,o2,o3}
				\fmf{gluon, tension=0.7}{i1,v1}
				\fmf{gluon, tension=0.7}{i3,v3}
				\fmf{fermion}{o1,v1}
				\fmf{fermion}{v1,v3}
				\fmf{fermion}{v3,v2}
				\fmf{fermion}{v2,o3}
				\fmf{phantom, tension=0.5}{i2,v1}
				\fmffreeze
				\fmf{dashes}{v2,o2}
				\fmflabel{$g$}{i1}
				\fmfv{l=$g$,l.a=-140}{i3}
				\fmflabel{$H$}{o2}
				\fmflabel{$t$}{o3}
				\fmflabel{$\bar{t}$}{o1}
			\end{fmfgraph*}
		\end{fmffile}
	}~~~~~~~~~~
	\subfigure[$q\bar{q}$ annihilation.]{
		\centering
		\begin{fmffile}{fgraphs2}
			\begin{fmfgraph*}(160,80)
				\fmfleft{i1,i2,i3,i4}
				\fmfright{o1,o2,o3}
				\fmf{fermion}{i4,v2,i1}
				\fmf{fermion}{o1,v1}
				\fmf{fermion}{v1,v3}
				\fmf{fermion}{v3,o3}
				\fmf{gluon}{v1,v2}
				\fmf{phantom}{o2,v1}
				\fmf{phantom}{v2,i2}
				\fmffreeze
				\fmf{dashes}{v3,o2}
				\fmflabel{$\bar{q}$}{i1}
				\fmflabel{$q$}{i4}
				\fmflabel{$H$}{o2}
				\fmflabel{$\bar{t}$}{o1}
				\fmflabel{$t$}{o3}
			\end{fmfgraph*}
		\end{fmffile}
	}
	\caption{Leading order $t\bar{t}$ production with Higgs radiation.}
	\label{fig:ttH}
	\end{center}
\end{figure}

The aim of my project was to develop the method in question by using semi-leptonic (SL) channel\footnote{Semi-leptonic (or single-leptonic) channel in the current context means that only one of the $W$-bosons resulting from $t\to W^+b\,$ ($\bar{t}\to W^-\bar{b}$), decays leptonically ($W^-\to \mathcal{\ell}\bar{\nu}_\mathcal{\ell}\,,\,\,W^+\to \bar{\mathcal{\ell}}\nu_\mathcal{\ell}$) and the other hadronically ($W\to q\bar{q'}$). Thus the final state of SL $t\bar{t}H$ has one tight lepton and 6 jets of which 4 are coming from $b$-quarks.}
of $t\bar{t}+\mbox{jets}$ MC simulations as a test bench.

\noindent The workflow went as follows:
\begin{enumerate}
\item Derivation of CSV tagger histograms (Fig. \ref{fig:hist_log_[40,60]_[0.8,1.6]}) for specific ranges of jet $p_t$ (transverse momentum of the jet), $|\eta|$ (pseudorapidity) and three flavours (quarks and antiquarks are treated in the same way).
The ranges are given in the table below:
\begin{center}
	\begin{tabular}{ c | c }
		\hline
		variable & ranges \\ \hline
		$p_t$ (GeV) & $[20,30),\,[30,40),\,[40,60),\,[60,100),\,[100,160),\,[160,\infty)$ \\ \hline
		$|\eta|$ & $[0,0.8),\,[0.8,1.6),\,[1.6,2.5)$ \\ \hline
		falvour & $b(\bar{b})$, $c(\bar{c})$, $g$ \\
	\end{tabular}
\end{center}
\item Sampling each histogram, i.e. random numbers were drawn from CSV distribution as many times as there were entries in it (see Fig. \ref{fig:hist_sampled_log_[40,60]_[0.8,1.6]}).
Kolmogorov-Smirnov and $\chi^2$ tests were carried out between the resulting histograms and sampled distributions.
Since all 54 histograms passed the tests, they were treated as probability distribution functions (PDFs) from this stage on.
\item Plotting the b-tagging efficiency and mistag rate of $c$- and light jets for each $p_t$ and $\eta$ range just to validate the setup.
\item Tossing random CSV values (according to PDFs) until CSV of the jet passed the standard $\mbox{CSVM}=0.679$ working point (i.e. is greater than or equal to CSVM).
The results were the same as if the PDFs were cut off at CSVM (Fig. \ref{fig:hist_multisampled_log_[40,60]_[0.8,1.6]}).
The number of iterations the jet needs to pass the working was also recorded for each jet.
Fig. \ref{fig:hist_iter_log_[60,100]_[0.8,1.6]} shows that the number of iterations falls of exponentially.
This is to be expected, because say if the probability that a given jet passes the working point is $p_i$, the probability of passing after $N$-th iteration is $(1-p_i)^{N-1}p_i$.
\item Only the events which contain exactly $N_{\text{jet}}$ jets with $p_t \geq 20$ GeV and $|\eta| < 2.5$ were selected.
While looping over the $N_{\text{iter}} = 10^4$ times, a CSV value was generated for each jet.
If the number of jets that passed CSVM cut coincides with the required number of b-tagged jets, $N_{\text{tag}}$, the number of passes, $N_{\text{passes}}$, was incremented.
The probability that a given $N_{\text{jet}}$-jet event contains $N_{\text{tag}}$ b-tagged jets is simply $P_M\left(N_{\text{tag}},\,N_{\text{jet}}\right)=\frac{N_{\text{passes}}}{N_{\text{iter}}}$.
\item The same events (that passed jet multiplicity and kinematical cuts) were once more sampled and their CSV values were recorded.
\item Derivation of cumulative histograms from PDFs in order to easily read the probability that a jet passes the CSVM cut (Fig. \ref{fig:cumul_[20,30]_[0.8,1.6]}).
While looping over the same jets as in 5. a combined probability was assigned to each event.
For example, if $N_{\text{jet}} = 3$, $N_{\text{tag}}=2$ and individual probabilities that a jet passes the working point are $p_1,\,p_2$ and $p_3$, the overall probability that an event passes the cut is
\[
P_A\left(N_{\text{tag}} = 3,\,N_{\text{jet}} = 2\right) = p_1 \cdot p_2 \cdot (1 - p_3) + p_2 \cdot p_3 \cdot (1 - p_1) + p_1 \cdot p_3 \cdot (1 - p_2)\,.
\]
The code works for any $N_{\text{jet}$ and $N_{\text{tag}$.
\item All events of interest passed the probability normalization check,
\[
\sum_{N_{\text{tag}} = 0}^{N_{\text{jet}}}P_{i}\left(N_{\text{tag}},\,N_{\text{jet}}) = 1\,,\quad i = A,M\,.
\]
Also, the probabilities for each event obtained in 5. and 7. were really close, although the possible values in 7. are limited.
\item Looping over all events and summing the probabilities found in 5. and 7. indicates that the integrals were nearly exact:
\[
\sum_{\text{events}}P_{M}\left(N_{\text{tag}},\,N_{\text{jet}}\right) \approx \sum_{\text{events}}P_{A}\left(N_{\text{tag}},\,N_{\text{jet}})\,.
\]
By looking at the randomly generated CSV values in 6., there was up to 2\% of discrepancy between the number of events in which the number of b-tagged jets equals to $N_{\text{tag}}$, and the sum of probabilities $\sum P_i\,\,(i = A,\,M)$.
\item Jet $p_t$, $|\eta|$ and measured CSV were plotted with both of the following methods (Fig. \ref{fig:jet_vars}):
\begin{enumerate}
\item loop over all $N_{\text{jet}}$ jets, read generated CSV value and check if the number of b-tags in an event equals to $N_{\text{tag}}$; if so, fill the histogram (the so-called hard cut method);
\item loop over all $N_{\text{jet}}$ jets and fill the histogram with weight $P_{i}\left(N_{\text{tag}},\,N_{\text{jet}}\right)$ with $i = A,\,M$ (reweighting method).
\end{enumerate}
Since the two histograms passed Kolmogorov-Smirnov and $\chi^2$ tests, it is reasonable to only use analytical probabilities as weights because method 5. is computationally intensive.
\end{enumerate}
In conclusion, the proposed reweighting method works with CSV distributions of specific $p_t$, $|\eta|$ and flavor ranges.
All histograms and events passed statistical tests and about 99\% of the events were saved in case of $N_{\text{tag}}=4,\,N_{\text{jet}}=6$ (see Table \ref{tab:percentage}).
The analysis was done without the lepton cuts.
As stated above, the method is inapplicable to data from measurements because one must know the jet flavour in the first place.
\vspace*{3em}
\section*{Appendix}
\begin{figure}[H]
\subfigure[CSV PDF derived from MC simulations.]{
	\centering
	\includegraphics[scale=.4]{../plots/eps/hist_log_[100,160]_[1.6,2.5].eps}
	\label{fig:hist_log_[40,60]_[0.8,1.6]}
}
\subfigure[Cumulative distribution function (CDF) of the same PDF.]{
	\centering
	\includegraphics[scale=.4]{../plots/cumul/cumul_[100,160]_[1.6,2.5].eps}
	\label{fig:cumul_[20,30]_[0.8,1.6]}
}

\subfigure[Sampled CSV distribution.]{
	\centering
	\includegraphics[scale=.4]{../plots/eps/hist_sampled_log_[100,160]_[1.6,2.5].eps}
	\label{fig:hist_sampled_log_[40,60]_[0.8,1.6]}
}
\subfigure[Sampled CSV distribution until the jet passes the working point. All events are included.]{
	\centering
	\includegraphics[scale=.4]{../plots/eps/hist_multisampled_log_[100,160]_[1.6,2.5].eps}
	\label{fig:hist_multisampled_log_[40,60]_[0.8,1.6]}
}

\subfigure[Number of iterations needed to pass the working point.]{
	\centering
	\includegraphics[scale=.4]{../plots/eps/hist_iter_log_[100,160]_[1.6,2.5].eps}
	\label{fig:hist_iter_log_[60,100]_[0.8,1.6]}
}

\caption{Results obtained from 1.-4.}
\end{figure}

\begin{figure}[H]
\subfigure[Jet $p_t$.]{
	\centering
	\includegraphics[scale=.4]{../plots/6_4/pt.eps}
	\label{fig:pt}
}
\subfigure[Jet $\eta$.]{
	\centering
	\includegraphics[scale=.4]{../plots/6_4/eta.eps}
	\label{fig:eta}
}

\subfigure[Measured jet CSV.]{
	\centering
	\includegraphics[scale=.4]{../plots/6_4/csv.eps}
	\label{fig:csv}
}
\caption{Hard cuts vs reweighted events ($N_{\text{tag}}=4,\,N_{\text{jet}}=6$).}
\label{fig:jet_vars}
\end{figure}
%\begin{table}[H]
%	\begin{center}
%		\begin{tabular}{ c | c | c }
%			\hline
%			Kinematic variable & \mcell{Kolmogorov-Smirnov\\test statistic} & \mcell{$\chi^2$ test\\$p$-value} \\ \hline
%			CSV & 0.501675 & 0.998086 \\
%			$\eta$ & 0.765578 & 0.998891 \\
%			$p_t$ & 0.999045 & 0.999999 \\
%		\end{tabular}
%	\end{center}
%	\caption{Statistical tests between histograms ($N_{\text{tag}}=4,\,N_{\text{jet}}=6$).}
%\end{table}
\begin{table}[H]
	\begin{center}
		\begin{tabular}{ c | c | c }
			\hline
			$N_{\text{tag}}$ & Number of events & Percentage of all events \\ \hline
			0 & 319392 & 13.81\% \\
			1 & 992953 & 42.95\% \\
			2 & 850589 & 36.79\% \\
			3 & 138667 & 5.99\% \\
			4 & 9566 & 0.413\% \\
			5 & 351 & 0.015\% \\
			6 & 4 & $\sim$0\% \\ \hline
			total & 2311522 & 100\%
		\end{tabular}
	\end{center}
	\caption{Percentage of $N_{\text{tag}}$ b-tagged events for $N_{\text{jet}}=6$.}
	\label{tab:percentage}
\end{table}

% here be references
\printbibliography

\end{document}
