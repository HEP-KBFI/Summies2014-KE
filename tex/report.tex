\documentclass[12pt,a4paper]{article}

% paketid
\usepackage[utf8]{inputenc}
\usepackage[english]{babel}
\usepackage{fullpage}
\usepackage[raggedright]{titlesec}
\usepackage[titles,subfigure]{tocloft}
\usepackage{hyperref}
\usepackage[svgnames]{xcolor}
\usepackage{mathtools}
\usepackage{float}
\usepackage[pdftex]{graphicx}
\usepackage[titletoc]{appendix}
\usepackage{enumerate}
\usepackage{calrsfs}
\usepackage{times}
\usepackage{titlesec}
\usepackage[letterspace=350]{microtype} % no hyphenation anymore
\usepackage[shortlabels]{enumitem}
\usepackage[bottom]{footmisc}
%\usepackage[margin=0.7in]{geometry}
\usepackage{multicol}
\usepackage{subfigure}
\usepackage{braket}
\def\pdfshellescape{1}
\usepackage{epstopdf}
\usepackage{wrapfig}
\usepackage{setspace}
\usepackage[multidot]{grffile}
\usepackage{feynmp} % Feynman diagrams
%\usepackage[backend=bibtex,sorting=none,style=ieee]{biblatex}

%\addbibresource{sources.bib}

%\linespread{1.5} % reavahe
\titleformat{\chapter}[hang]{\bfseries\Large}{\thechapter.}{2pc}{} % teeb pealkirjad korda
\titlelabel{\thetitle.\quad}
\titlespacing*{\chapter}{0pt}{-50pt}{20pt}
\titlespacing*{\section}{0pt}{-15pt}{5pt}
\titlespacing*{\subsection}{0pt}{0pt}{0pt}
\setlist[enumerate]{itemsep=0mm}

\subfigcapskip = 20pt

\definecolor{dark-red}{rgb}{0.4,0.15,0.15}

\hypersetup{
	colorlinks=true,
    linkcolor={Black},
    citecolor={Black}, urlcolor={dark-red}
}

\newcommand{\mcell}[2][c]{
	\begin{tabular}[#1]{@{}c@{}}#2\end{tabular}
}

\DeclareMathAlphabet{\pazocal}{OMS}{zplm}{m}{n}

\DeclareGraphicsExtensions{.pdf,.png,.jpg,.eps}
\DeclareGraphicsRule{*}{mps}{*}{}

%\setlength\parindent{0pt} % noindent by default

\newcommand*{\TitleFont}{%
      \usefont{\encodingdefault}{\rmdefault}{b}{n}%
      \fontsize{16}{20}%
      \selectfont}

%\DeclareMathAlphabet{\pazocal}{OMS}{zplm}{m}{n} % teistsugune kalligraafiline täht, nt \pazocal{L}

\epstopdfsetup{outdir=./converted_pdf/}

\title{\TitleFont\textbf{Report on the Summer Student project\\,,Improved background estimation for the $t\bar{t}H$ analysis''}}
\author{Karl Ehatäht\\[0.5em]
\textit{Supervisors: Lorenzo Bianchini, Mario Kadastik, Joosep Pata}}
\date\today

\begin{document}
\maketitle

A direct probe of the Yukawa coupling of Higgs boson to top quark can be measured in only those processes where top quark radiates a Higgs boson (see Fig. \ref{fig:ttH}).
The process, denoted as $t\bar{t}H$, has the largest cross section when the Higgs decay mode, $H \to b\bar{b}$, is the most dominant.
The large multijet background in the final state  can be reduced by selecting only the events with one or two charged isolated leptons.
However, the signal is still plauged by QCD production of $t\bar{t}+b\bar{b}$, $t\bar{t}+c\bar{c}$ and $t\bar{t}+jj$ after the cut.
Although the former subprocess, $pp\to t\bar{t}b\bar{b}$, remains still irreducible, the rest can be suppressed by requiring the jets to be b-tagged.
In other words, a real number (discriminator) is calculated (based on the jet kinematics and information about secondary vertex) and then assigned to each jet which is used as a boolean variable to separate $b$-jets from non-$b$ jets, as it is usually carried out in standard analysis.
B-tagging efficiency and mistag rate of $c$- and light jets depend on the b-tagging algorithm, e.g. CSV (combined secondary vertex).
Since the distribution of discriminator is known from Monte Carlo (MC) simulations, it can be used probabilistically to reweigh the events.
The method rescues all MC events instead of throwing most of them away.
Intrinsically, as it turns out, the method cannot be applied on data from measurements.
\vspace*{1em}
\begin{figure}[H]
	\begin{center}
	\subfigure[t-channel gluon fusion.]{
		\centering
		\begin{fmffile}{fgraphs}
			\begin{fmfgraph*}(160,80)
				\fmfleft{i1,i2,i3}
				\fmfright{o1,o2,o3}
				\fmf{gluon, tension=0.7}{i1,v1}
				\fmf{gluon, tension=0.7}{i3,v3}
				\fmf{fermion}{o1,v1}
				\fmf{fermion}{v1,v3}
				\fmf{fermion}{v3,v2}
				\fmf{fermion}{v2,o3}
				\fmf{phantom, tension=0.5}{i2,v1}
				\fmffreeze
				\fmf{dashes}{v2,o2}
				\fmflabel{$g$}{i1}
				\fmfv{l=$g$,l.a=-140}{i3}
				\fmflabel{$H$}{o2}
				\fmflabel{$t$}{o3}
				\fmflabel{$\bar{t}$}{o1}
			\end{fmfgraph*}
		\end{fmffile}
	}~~~~~~~~~~
	\subfigure[$q\bar{q}$ annihilation.]{
		\centering
		\begin{fmffile}{fgraphs2}
			\begin{fmfgraph*}(160,80)
				\fmfleft{i1,i2,i3,i4}
				\fmfright{o1,o2,o3}
				\fmf{fermion}{i4,v2,i1}
				\fmf{fermion}{o1,v1}
				\fmf{fermion}{v1,v3}
				\fmf{fermion}{v3,o3}
				\fmf{gluon}{v1,v2}
				\fmf{phantom}{o2,v1}
				\fmf{phantom}{v2,i2}
				\fmffreeze
				\fmf{dashes}{v3,o2}
				\fmflabel{$\bar{q}$}{i1}
				\fmflabel{$q$}{i4}
				\fmflabel{$H$}{o2}
				\fmflabel{$\bar{t}$}{o1}
				\fmflabel{$t$}{o3}
			\end{fmfgraph*}
		\end{fmffile}
	}
	\caption{Leading order $t\bar{t}$ production with Higgs radiation.}
	\label{fig:ttH}
	\end{center}
\end{figure}

The aim of my project was to develop the method in question by using semi-leptonic (SL) channel\footnote{Semi-leptonic (or single-leptonic) channel in current context means that only one of the $W$-bosons resulting from $t\to W^+b\,$ ($\bar{t}\to W^-\bar{b}$), decays leptonically ($W^-\to \mathcal{\ell}\bar{\nu}_\mathcal{\ell}\,,\,\,W^+\to \bar{\mathcal{\ell}}\nu_\mathcal{\ell}$) and the other hadronically ($W\to q\bar{q'}$). Thus the final state of SL $t\bar{t}H$ has one tight lepton and 6 jets of which 4 are coming from $b$-quarks.}
of $t\bar{t}+\mbox{jets}$ MC simulations as a test bench.\\

\noindent The workflow went as follows:
\begin{enumerate}
\item a
\end{enumerate}


\end{document}
